%%This is a very basic article template.
%%There is just one section and two subsections.
\documentclass{article}

\begin{document}


\section{Polynomes}
$x$ une variable (element inconnu) , $n \in N$  $a_0,a_1,a_2,\ldots a_n$ sont
des nombres reels
$P(x) = a_nx^n+a_{n-1}x^{n-1}+ \ldots + a_2x^2+ a_1x + a_0$ est l'expression
d'un polynome a une variable reelle.
%%\subsection{Subtitle}
\paragraph{Pour ce polynome:}
\begin{itemize}
\item[$\ast$] $a_nx^n+a_{n-1}x^{n-1}+ \ldots + a_2x^2+ a_1x + a_0$ sont les
termes du polynome.
\item[$\ast$] $a_n,a_{n-1}, \ldots , a_2, a_1 , a_0$ sont $(n+1)$ reels appeles
les coefficients du polynome.
\item[$\ast$] $a_n$ est le coefficient principal du polynome (coefficient du
terme de plus haut degre).
\item[$\ast$] $a_0$ est le terme constant du polynome.
\item[$\ast$] $P(x)$ est le polynome donne ci-dessus, n est le degre du polynome
(an\#$0$) On note: $der[P(x)]=n$
\item[$\ast$]$P(x, y)$ est l'expression d'un polynome a 2 variables $P(x,y,z)$
est l'expression d'un polynome a 3 variables.
\end{itemize}
\subparagraph{Polynome Constant:}
$P(x=a_0)$ est le polynome constant. Son degre est 0. $der[P(x)]=0$.
\subparagraph{Polynome Nul:}
$P(x)=0$ est le polynome nul. Son degre est.
\begin{center}
    \begin{tabular}{ | l | l | l | l | l | l |}
    \hline
    Function & Polynome? & Deure? & Coorricent principal & Terme Constant &
    Somme Des Cofficients\\ \hline 
    $f(x)=x^3-x^2+2$ & O & 3 & 1 & 2 & 2\\ \hline
    $f(x)=x^2-3x^3+\frac{1}{2}$ & O & $3$ & $-3$ & $\frac{1}{2}$ &
    $-\frac{3}{2}$ \\ \hline 
    $f(x)=\frac{1}{x}+2$  & N & - & - & - & - \\ \hline
    $f(x)=x+x^2-\sqrt{x}$  & N & - & - & - & - \\ \hline
    $f(x)=\frac{2}{3}$  & O & $0$ & $\frac{2}{3}$ & $\frac{2}{3}$ &
    $\frac{2}{3}$
    \\
    \hline
    
    \end{tabular}
\end{center}
\end{document}
