\usepackage[english]{babel}
\usepackage[latin5]{inputenc}

\usepackage{listings}
\lstset{
  literate={~} {$\sim$}{1}, % set tilde as a literal (no process)      % the language of the code
  basicstyle=\ttfamily\fontsize{7}{8}\selectfont\bfseries,
  frame=tb,
  linewidth=0.98\linewidth,
  columns=flexible,
  numbers=left,                   % where to put the line-numbers
  stepnumber=1,                   % the step between two line-numbers. If it's 1, each line will be numbered
  numbersep=5pt,                  % how far the line-numbers are from the code
  showspaces=false,               % show spaces adding particular underscores
  showstringspaces=false,         % underline spaces within strings
  showtabs=false,                 % show tabs within strings adding particular underscores
  frame=single,                   % adds a frame around the code
  tabsize=2,                      % sets default tabsize to 2 spaces
  captionpos=b,                   % sets the caption-position to bottom
  breaklines=true,                % sets automatic line breaking
  breakatwhitespace=false,        % sets if automatic breaks should only happen at whitespace
  caption=\lstname,                 % show the filename of files included with \lstinputlisting; also try caption instead of title
  escapeinside={\%*}{*)},            % if you want to add a comment within your code
  morekeywords={*,...},               % if you want to add more keywords to the set
  lineskip={-1.5pt}
}
% 


\renewcommand{\lstlistingname}{Code}
\renewcommand{\lstlistlistingname}{List of Codes}

\usepackage{latexsym}
\usepackage{graphicx}
\usepackage{amsfonts}
\usepackage{amssymb}
\usepackage{fancyhdr}
\usepackage{setspace}
\usepackage{float}
\usepackage{multicol}


% \usepackage{listings}
%\faculty{M�hendislik Fak�ltesi Bilgisayar M�hendisli�i B�l�m�}
\usepackage{url}

\tolerance=10000
\hyphenpenalty=5000

