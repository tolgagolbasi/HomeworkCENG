\section{Introduction to GMP}
GNU MP is a portable library written in C for arbitrary precision arithmetic on
integers, rational numbers, and floating-point numbers. It aims to provide the
fastest possible arithmetic for all applications that need higher precision than
is directly supported by the basic C types.

Many applications use just a few hundred bits of precision; but some
applications may need thousands or even millions of bits. GMP is designed to
give good performance for both, by choosing algorithms based on the sizes of the
operands, and by carefully keeping the overhead at a minimum.

The speed of GMP is achieved by using fullwords as the basic arithmetic type, by
using sophisticated algorithms, by including carefully optimized assembly code
for the most common inner loops for many different CPUs, and by a general
emphasis on speed (as opposed to simplicity or elegance).

There is assembly code for these CPUs: ARM Cortex-A9, Cortex-A15, and generic
ARM, DEC Alpha 21064, 21164, and 21264, AMD K8 and K10 (sold under many brands,
e.g. Athlon64, Phenom, Opteron) Bulldozer, and Bobcat, Intel Pentium, Pentium
Pro/II/III, Pentium 4, Core2, Nehalem, Sandy bridge, Haswell, generic x86, Intel
IA-64, Motorola/IBM PowerPC 32 and 64 such as POWER970, POWER5, POWER6, and
POWER7, MIPS 32-bit and 64-bit, SPARC 32-bit ad 64-bit with special support for
all UltraSPARC models. There is also assembly code for many obsolete CPUs.

For up-to-date information on GMP, please see the GMP web pages at

     https://gmplib.org/
The latest version of the library is available at

     https://ftp.gnu.org/gnu/gmp/

\section{GMP Basics}
3.1 Headers and Libraries

All declarations needed to use GMP are collected in the include file gmp.h. It
is designed to work with both C and C++ compilers.
\begin{lstlisting}
     #include <gmp.h>
\end{lstlisting}     
Note however that prototypes for GMP functions with FILE * parameters are only
provided if <stdio.h> is included too.
\begin{lstlisting}
     #include <stdio.h>
     #include <gmp.h>
\end{lstlisting}     
All programs using GMP must link against the libgmp library. On a typical
Unix-like system this can be done with '-lgmp', for example gcc myprogram.c
-lgmp

3.2 Nomenclature and Types

Integer usually means a multiple precision integer, as defined
by the GMP library. The C data type for such integers is mpz\_t. Here are some
examples of how to declare such integers:
\begin{lstlisting}
     mpz_t sum;
     
     struct foo { mpz_t x, y; };
     
     mpz_t vec[20];
\end{lstlisting}         

A limb means the part of a multi-precision number that fits in a single machine
word. Normally a limb is 32 or 64 bits. The C data type for a limb is
mp\_limb\_t.

Counts of limbs of a multi-precision number represented in the C type mp\_size\_t.

Also, in general mp\_bitcnt\_t is used for bit counts and ranges, and size\_t is
used for byte or character counts.

3.3 Function Classes

These are some of classes of functions in the GMP library:

Functions for signed integer arithmetic, with names beginning with mpz\_. The
associated type is mpz\_t. There are about 150 functions in this class.
Functions for rational number arithmetic, with names beginning with mpq\_. The
associated type is mpq\_t. There are about 35 functions in this class, but the
integer functions can be used for arithmetic on the numerator and denominator
separately.
Functions for floating-point arithmetic, with names beginning with mpf\_. The
associated type is mpf\_t. There are about 70 functions is this class.
Fast low-level functions that operate on natural numbers. These functions' names
begin with mpn\_. The associated type is array of mp\_limb\_t. There are about
60 (hard-to-use) functions in this class.
Miscellaneous functions. Functions for setting up custom allocation and
functions for generating random numbers.

3.4 Variable Conventions

GMP functions generally have output arguments before input arguments. This
notation is by analogy with the assignment operator.

GMP lets you use the same variable for both input and output in one call. For
example, the main function for integer multiplication, mpz\_mul, can be used to
square x and put the result back in x with

     mpz\_mul (x, x, x);
Before you can assign to a GMP variable, you need to initialize it by calling
one of the special initialization functions. When you're done with a variable,
you need to clear it out, using one of the functions for that purpose. Which
function to use depends on the type of variable.

A variable should only be initialized once, or at least cleared between each
initialization. After a variable has been initialized, it may be assigned to any
number of times.

For efficiency reasons, avoid excessive initializing and clearing. In general,
initialize near the start of a function and clear near the end. For example,
\begin{lstlisting}
     void
     foo (void)
     {
       mpz_t  n;
       int    i;
       mpz_init (n);
       for (i = 1; i < 100; i++)
         {
           mpz_mul (n, ...);
           mpz_fdiv_q (n, ...);
           ...
         }
       mpz_clear (n);
     }
\end{lstlisting} 
3.5 Parameter Conventions

When a GMP variable is used as a function parameter, it's effectively a
call-by-reference, meaning if the function stores a value there it will change
the original in the caller.

When a function is going to return a GMP result, it should designate a
parameter that it sets, like the library functions do. More than one value can
be returned by having more than one output parameter, again like the library
functions. A return of an mpz\_t etc doesn't return the object, only a pointer,
and this is almost certainly not what's wanted.

Here's an example accepting an mpz\_t parameter, doing a calculation, and
storing the result to the indicated parameter.
\begin{lstlisting}
     void
     foo (mpz_t result, const mpz_t param, unsigned long n)
     {
       unsigned long  i;
       mpz_mul_ui (result, param, n);
       for (i = 1; i < n; i++)
         mpz_add_ui (result, result, i*7);
     }

     int
     main (void)
     {
       mpz_t  r, n;
       mpz_init (r);
       mpz_init_set_str (n, "123456", 0);
       foo (r, n, 20L);
       gmp_printf ("%Zd\n", r);
       return 0;
     }
\end{lstlisting} 
3.6 Memory Management

The GMP types like mpz\_t are small, containing only a couple of sizes, and
pointers to allocated data. Once a variable is initialized, GMP takes care of
all space allocation. Additional space is allocated whenever a variable doesn't
have enough.

mpz\_t and mpq\_t variables never reduce their allocated space. Normally this is
the best policy, since it avoids frequent reallocation. Applications that need
to return memory to the heap at some particular point can use mpz\_realloc2, or
clear variables no longer needed.

All memory is allocated using malloc and friends by default, but this can be
changed, see Custom Allocation. Temporary memory on the stack is also used (via
alloca), but this can be changed at build-time if desired, see Build Options.

3.11 Efficiency

Small Operands
On small operands, the time for function call overheads and memory allocation
can be significant in comparison to actual calculation. This is unavoidable in a
general purpose variable precision library, although GMP attempts to be as
efficient as it can on both large and small operands.

Initializing and Clearing
Avoid excessive initializing and clearing of variables, since this can be quite
time consuming, especially in comparison to otherwise fast operations like
addition.

Reallocations
An mpz\_t or mpq\_t variable used to hold successively increasing values will
have its memory repeatedly realloced, which could be quite slow or could
fragment memory, depending on the C library. If an application can estimate the
final size then mpz\_init2 or mpz\_realloc2 can be called to allocate the
necessary space from the beginning (see Initializing Integers).
It doesn't matter if a size set with mpz\_init2 or mpz\_realloc2 is too small,
since all functions will do a further reallocation if necessary. Badly
overestimating memory required will waste space though.

mpz\_mul is currently the opposite, a separate destination is slightly better. A
call like mpz\_mul(x,x,y) will, unless y is only one limb, make a temporary copy
of x before forming the result. Normally that copying will only be a tiny
fraction of the time for the multiply, so this is not a particularly important
consideration.

Divisibility Testing (Small Integers)
mpz\_divisible\_ui\_p and mpz\_congruent\_ui\_p are the best functions for
testing whether an mpz\_t is divisible by an individual small integer. They use
an algorithm which is faster than mpz\_tdiv\_ui, but which gives no useful
information about the actual remainder, only whether it's zero (or a particular
value).
However when testing divisibility by several small integers, it's best to take a
remainder modulo their product, to save multi-precision operations.

The division functions like mpz\_tdiv\_q\_ui which give a quotient as well as a
remainder are generally a little slower than the remainder-only functions like
mpz\_tdiv\_ui. If the quotient is only rarely wanted then it's probably best to
just take a remainder and then go back and calculate the quotient if and when
it's wanted (mpz\_divexact\_ui can be used if the remainder is zero). 

\section{Installing GMP}
GMP has an autoconf/automake/libtool based configuration system. On a Unix-like
system a basic build can be done with

     ./configure
     make
Some self-tests can be run with

     make check
And you can install (under /usr/local by default) with

     make install

CPU types
In general, if you want a library that runs as fast as possible, you should
configure GMP for the exact CPU type your system uses. However, this may mean
the binaries won't run on older members of the family, and might run slower on
other members, older or newer. The best idea is always to build GMP for the
exact machine type you intend to run it on.
 
Generic C Build
If some of the assembly code causes problems, or if otherwise desired, the
generic C code can be selected with the configure --disable-assembly.
Note that this will run quite slowly, but it should be portable and should at
least make it possible to get something running if all else fails.

Fat binary, --enable-fat
Using --enable-fat selects a fat binary build on x86, where optimized low level
subroutines are chosen at runtime according to the CPU detected. This means more
code, but gives good performance on all x86 chips.

FFT Multiplication, --disable-fft
By default multiplications are done using Karatsuba, 3-way Toom, higher degree
Toom, and Fermat FFT. The FFT is only used on large to very large operands and
can be disabled to save code size if desired.

In particular for long-running GMP applications, and applications demanding
extremely large numbers, building and running the tuneup program in the tune
subdirectory, can be important. For example,

     cd tune
     make tuneup
     ./tuneup
will generate better contents for the gmp-mparam.h parameter file.

\section{Integer Functions}
The functions for integer arithmetic assume that all integer objects are
initialized. You do that by calling the function mpz\_init. For example,
\begin{lstlisting}
     {
       mpz_t integ;
       mpz_init (integ);
       ...
       mpz_add (integ, ...);
       ...
       mpz_sub (integ, ...);
     
       /* Unless the program is about to exit, do ... */
       mpz_clear (integ);
     }
\end{lstlisting}     
As you can see, you can store new values any number of times, once an object is
initialized.

 Function: void mpz\_init (mpz\_t x)
Initialize x, and set its value to 0.

 Function: void mpz\_inits (mpz\_t x, ...)
Initialize a NULL-terminated list of mpz\_t variables, and set their values to
0.

 Function: void mpz\_clear (mpz\_t x)
Free the space occupied by x. Call this function for all mpz\_t variables when
you are done with them.

 Function: void mpz\_clears (mpz\_t x, ...)
Free the space occupied by a NULL-terminated list of mpz\_t variables.

 Function: void mpz\_realloc2 (mpz\_t x, mp\_bitcnt\_t n)
Change the space allocated for x to n bits. The value in x is preserved if it
fits, or is set to 0 if not.

These functions assign new values to already initialized integers (see
Initializing Integers).

 Function: void mpz\_set (mpz\_t rop, const mpz\_t op)
Set the value of rop from op.

 Function: int mpz\_set\_str (mpz\_t rop, const char *str, int base)
Set the value of rop from str, a null-terminated C string in base base.

 Function: void mpz\_swap (mpz\_t rop1, mpz\_t rop2)
Swap the values rop1 and rop2 efficiently.

Here is an example of using one:
\begin{lstlisting}
     {
       mpz_t pie;
       mpz_init_set_str (pie, "3141592653589793238462643383279502884", 10);
       ...
       mpz_sub (pie, ...);
       ...
       mpz_clear (pie);
     }
\end{lstlisting}
Once the integer has been initialized by any of the mpz\_init\_set... functions,
it can be used as the source or destination operand for the ordinary integer
functions.

 Function: void mpz\_init\_set (mpz\_t rop, const mpz\_t op)
Initialize rop with limb space and set the initial numeric value from op.

 Function: int mpz\_init\_set\_str (mpz\_t rop, const char *str, int base)
Initialize rop and set its value like mpz\_set\_str

 Function: unsigned long int mpz\_get\_ui (const mpz\_t op)
Return the value of op as an unsigned long.

 Function: signed long int mpz\_get\_si (const mpz\_t op)
If op fits into a signed long int return the value of op. Otherwise return the
least significant part of op, with the same sign as op.

 Function: char * mpz\_get\_str (char *str, int base, const mpz\_t op)
Convert op to a string of digits in base base.

5.5 Arithmetic Functions

 Function: void mpz\_add (mpz\_t rop, const mpz\_t op1, const mpz\_t op2)
Set rop to op1 + op2.

 Function: void mpz\_sub (mpz\_t rop, const mpz\_t op1, const mpz\_t op2)
Set rop to op1 - op2.

 Function: void mpz\_mul (mpz\_t rop, const mpz\_t op1, const mpz\_t op2)
Set rop to op1 times op2.

 Function: void mpz\_addmul (mpz\_t rop, const mpz\_t op1, const mpz\_t op2)
Set rop to rop + op1 times op2.

 Function: void mpz\_submul (mpz\_t rop, const mpz\_t op1, const mpz\_t op2)
Set rop to rop - op1 times op2.

 Function: void mpz\_mul\_2exp (mpz\_t rop, const mpz\_t op1, mp\_bitcnt\_t op2)
Set rop to op1 times 2 raised to op2. This operation can also be defined as a
left shift by op2 bits.

 Function: void mpz\_neg (mpz\_t rop, const mpz\_t op)
Set rop to -op.

 Function: void mpz\_abs (mpz\_t rop, const mpz\_t op)
Set rop to the absolute value of op.

 Function: void mpz\_mod (mpz\_t r, const mpz\_t n, const mpz\_t d)
Set r to n mod d. The sign of the divisor is ignored; the result is always
non-negative.

These routines are much faster than the other division functions, and are the
best choice when exact division is known to occur, for example reducing a
rational to lowest terms.

 Function: int mpz\_divisible\_p (const mpz\_t n, const mpz\_t d)
Return non-zero if n is exactly divisible by d, or in the case of
mpz\_divisible\_2exp\_p by 2\^b.

n is divisible by d if there exists an integer q satisfying n = q*d. Unlike the
other division functions, d=0 is accepted and following the rule it can be seen
that only 0 is considered divisible by 0.

 Function: int mpz\_congruent\_p (const mpz\_t n, const mpz\_t c, const mpz\_t d)
Return non-zero if n is congruent to c modulo d, or in the case of
mpz\_congruent\_2exp\_p modulo 2\^b.

n is congruent to c mod d if there exists an integer q satisfying n = c + q*d.
Unlike the other division functions, d=0 is accepted and following the rule it
can be seen that n and c are considered congruent mod 0 only when exactly equal.

 Function: void mpz\_sqrt (mpz\_t rop, const mpz\_t op)
Set rop to the truncated integer part of the square root of op.

 Function: int mpz\_cmp (const mpz\_t op1, const mpz\_t op2)
Compare op1 and op2. Return a positive value if op1 > op2, zero if op1 = op2, or
a negative value if op1 < op2.

 Macro: int mpz\_sgn (const mpz\_t op)
Return +1 if op > 0, 0 if op = 0, and -1 if op < 0.

5.12 Input and Output Functions

 Function: size\_t mpz\_out\_str (FILE *stream, int base, const mpz\_t op)
Output op on stdio stream stream, as a string of digits in base base. The base
argument may vary from 2 to 62 or from -2 to -36.

 Function: size\_t mpz\_inp\_str (mpz\_t rop, FILE *stream, int base)
Input a possibly white-space preceded string in base base from stdio stream
stream, and put the read integer in rop.

Function: size\_t mpz\_out\_raw (FILE *stream, const mpz\_t op)
Output op on stdio stream stream, in raw binary format. The integer is written
in a portable format, with 4 bytes of size information, and that many bytes of
limbs. Both the size and the limbs are written in decreasing significance order
(i.e., in big-endian).

Function: size\_t mpz\_inp\_raw (mpz\_t rop, FILE *stream)
Input from stdio stream stream in the format written by mpz\_out\_raw, and put
the result in rop.


5.14 Integer Import and Export

mpz\_t variables can be converted to and from arbitrary words of binary data
with the following functions.

 Function: void mpz\_import (mpz\_t rop, size\_t count, int order, size\_t size,
 int endian, size\_t nails, const void *op) Set rop from an array of word data at op.

The parameters specify the format of the data. count many words are read, each
size bytes. order can be 1 for most significant word first or -1 for least
significant first. Within each word endian can be 1 for most significant byte
first, -1 for least significant first, or 0 for the native endianness of the
host CPU. The most significant nails bits of each word are skipped, this can be
0 to use the full words.

Here's an example converting an array of unsigned long data, most significant
element first, and host byte order within each value.
\begin{lstlisting}
          unsigned long  a[20];
      /* Initialize z and a */
      mpz\_import (z, 20, 1, sizeof(a[0]), 0, 0, a);
\end{lstlisting}      

 Function: void * mpz\_export (void *rop, size\_t *countp, int order, size\_t size, int endian, size\_t nails, const mpz\_t op)
Fill rop with word data from op.

The parameters specify the format of the data produced. Each word will be size
bytes and order can be 1 for most significant word first or -1 for least
significant first. Within each word endian can be 1 for most significant byte
first, -1 for least significant first, or 0 for the native endianness of the
host CPU. The most significant nails bits of each word are unused and set to
zero, this can be 0 to produce full words. 
