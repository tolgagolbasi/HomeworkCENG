
\section{Motivation}
In a world where information getting more precious day by day vital part of the
security of the information is cryptographic systems. As a student of Cyber
Security Master's Program i wanted to learn how i can implement such systems
myself using world-wide accepted tools. I have choosen Elliptic Curve Method
because of its parallelizable first stage and further improvements coming from
better implementation of it. 
Also using GMP library and C programming with using a software versioning and
revision control system such as SVN gave me a real world idea of working on
projects other than loose schoolwork.
Furthermore working with Elliptic Curve Cryptography gave me the understanding
of working with coding tools that are reliable and supported well by the
community and companies. At start of my work i started working with OpenCL
graphics programming language. But over time i realized that OpenCL wasn't well
supported as i thought it was and it was hard to get help online on my mistakes
or compilers shortcomings.

\section{Outline}
This thesis is divided into 7 chapters. Chapter 1 gives an introduction and my
motivation for working for this thesis and also understanding about why
cryptographic systems are important and vital to our communities. Chapter 2
presents the foundations of well understood Elliptic Curve Method. Chapter 3 is
an introduction to using GMP library which used heavily in this thesis. Chapter
4 is focused on implementation of Elliptic Curve Method with regular curves and
improved version using Montgomery Curves. Chapter 5 is the section where
methodology and results of our implementation are given. In Chapter 6 we present
our results for testing the implementation. Chapter 7  summarizes the main
conclusions of the thesis and presents future work. 

\section{Introduction / Background}
Factorization problem is the  problem of finding factors of numbers. If you
are not familiar with factoring it might sound pretty easy for the
numbers we work with in our daily lives. But factorization of mostly the
big numbers are really hard problem to deal with. And these hardness today makes
us much more secure since a lot of cryptographic protocols depends on the
premise that no one can break this protocol by finding factors of our primes.
Earliest factoring aids were tables of prime factors
published back of math texts in 1600s. For numbers out of scope of the tables
new techniques had to be developed.
Simplest factorization method is trial division. Also trial division works very
fast for huge percentage of numbers. Because most of the numbers has small prime
factors. 50 percentage of the numbers divided by 2! Of course trial division
method can be extremely slow for numbers with very high primes.
Fermat's Algorithm and Euler's Algorithm did further improvements on
factorization but it was obvious that factorization were still extremely hard
problem without the aid of machines. 
D.H. Lehmer built two machines. First one is called bicycle-chain sieve. The
machine consisted of bicycle chains and electrical switches when all the
electrical switches close at the same time it created a electrical circuit and a
solution was found. Second one was Photoelectric Number Sieve consisted of 16 mm
film with holes in them replacing the bicycle chains in the old model. Also in
those days it was a common practice to use computers for factorization in their
idle time. And factorization problems intensity on computers were useful for IBM
engineers as they found hardware problems on computers that standard tests could
not.
With the emergence of the public key cryptography research on factorization
intensified.
Pollard's rho method
Pollard's p-1 method
Sub exponential methods
Qua

\section{Aim and objectives}
Aim and objectives go here.
Mandatory section.

\section{Deliverables}
Deliverables go here.
Mandatory section.

\section{Roadmap}
Roadmap goes here. Explain how the thesis is structured.
Optional section.


















\begin{enumerate}   
	\item Item 1
	\item Item 2
	\item Item 3
\end{enumerate}